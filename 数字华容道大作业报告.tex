\documentclass{article}
\usepackage[utf8]{inputenc}
\usepackage{graphicx}
\usepackage{titlesec}
\usepackage{listings}
\usepackage{enumitem}
\usepackage{geometry}
\usepackage{ctex}
\usepackage{xcolor}
\geometry{a4paper, left=2cm, right=2cm, top=2cm, bottom=2cm}

\titleformat{\section}{\normalfont\Large\bfseries}{\S\thesection}{1em}{}
\titleformat{\subsection}{\normalfont\large\bfseries}{\S\thesubsection}{1em}{}

\lstset{
    language=Python,
    basicstyle=\ttfamily\small,
    keywordstyle=\color{blue},
    stringstyle=\color{red},
    commentstyle=\color{gray},
    numbers=left,
    numberstyle=\tiny\color{gray},
    stepnumber=1,
    numbersep=5pt,
    backgroundcolor=\color{white},
    showspaces=false,
    showstringspaces=false,
    showtabs=false,
    frame=single,
    tabsize=2,
    breaklines=true,
    breakatwhitespace=false,
}

\begin{document}

\begin{center}
    \huge{\textbf{《Python程序设计高阶》}}
    \\
    (2025-2026学年第1学期)
    \\
    \vspace{1cm}
    \huge{期末报告}
    \\
    \vspace{1cm}
    \large 题目:数字华容道分布式求解系统
    \\
    \vspace{0.5cm}
    \large 小组成员:[陈卓尔] [罗竣泽] [胡枷浩]
\end{center}

\section{选题内容}

本项目为《Python程序设计-高阶》期末大报告选题第6题:分布式数字华容道。数字华容道是经典益智游戏,通过平移方块让数字按顺序归位。本项目旨在通过多程序协作的方式解决该问题。

\textbf{角色分配:}
\begin{itemize}
    \item \textbf{人机界面 (UI)}: 负责显示游戏界面、展示每一步走法、步数计数及通信日志。
    \item \textbf{计算程序 (Solver)}: 共两个计算成员,通过局域网连接至UI,交替计算并告诉人机界面下一步走法。
\end{itemize}

\textbf{测试流程:}
1. 拷贝代表初始状态的文件到人机界面目录下。
2. 启动各自程序,后续工作由程序自动完成。
3. 如果无解或游戏结束,给出结果。

\section{设计过程与实践}

\subsection{系统架构}
系统采用 \textbf{C/S (Server/Client) 架构}。UI程序作为服务端监听端口,两个Solver程序作为客户端连接。采用 TCP Socket 进行实时双向通信。

\subsection{求解算法:IDA*}
为了在处理 $4 \times 4$ 等高难度棋盘时既保证解的最优性又节省内存,我们实现了 \textbf{IDA* (Iterative Deepning A*)} 算法:
\begin{itemize}
    \item \textbf{启发函数}:结合了曼哈顿距离 (Manhattan Distance) 和线性冲突 (Linear Conflict) 优化,大幅减少搜索节点数。
    \item \textbf{迭代加深}:通过不断增加深度阈值,避免了传统 A* 算法在大规模状态空间下内存耗尽的问题。
\end{itemize}

\subsection{分布式协作机制}
为了确保真正的分布式协作而非“打酱油”:
\begin{itemize}
    \item \textbf{任务轮询}:UI程序维护一个全局步数统计。当步数为偶数时,通知 Solver 1 行动;当步数为奇数时,通知 Solver 2 行动。
    \item \textbf{状态同步}:每当一方移动后,UI会将最新的棋盘状态同步广播给所有 Solver,确保各节点始终基于最新状态计算。
\end{itemize}

\section{核心代码实现}

\subsection{棋盘状态表示 (common.py)}
\begin{lstlisting}
@dataclass
class PuzzleState:
    board: List[List[int]]
    def move(self, direction: Direction):
        # 执行滑块移动逻辑
        ...
    def is_solvable(self) -> bool:
        # 基于逆序数和空位位置的复杂判断逻辑
        ...
\end{lstlisting}

\subsection{IDA* 搜索核心 (solver.py)}
\begin{lstlisting}
def solve_puzzle(state: PuzzleState):
    threshold = linear_conflict(state)
    while True:
        result = search(state, 0, threshold)
        if isinstance(result, list): return result
        if result == float('inf'): return None
        threshold = result
\end{lstlisting}

\subsection{网络通信 (solver_node.py)}
\begin{lstlisting}
def _handle_message(self, msg: Message):
    if msg.msg_type == MessageType.STATE:
        self.current_state = PuzzleState(msg.board_data)
    elif msg.msg_type == MessageType.YOUR_TURN:
        self._make_move() # 计算并发送下一步
\end{lstlisting}

\section{代码运行与输出结果}

本系统经过多轮测试,能够稳定处理不同阶数和难度的棋盘:

\subsection{典型测试案例分析}
\begin{itemize}
    \item \textbf{测试用例 4 (3x3)}:中等难度。Solver 1 和 Solver 2 交替进行了约 20 步操作,每步耗时均小于 0.01 秒,展示了极高的计算效率。
    \item \textbf{测试用例 2 (4x4)}:高难度。由于 $4 \times 4$ 状态空间巨大($10^{13}$ 以上),IDA* 算法在此发挥了关键作用。系统成功在 2 分钟内找到了最优解,步数在 50 步左右。
    \item \textbf{测试用例 5 (3x3 无解)}:系统在加载文件后立即通过逆序数校验,UI 实时显示“题目无解”并同步给所有 Solver,符合作业要求的边界处理。
\end{itemize}

\subsection{运行环境及步骤}
\begin{enumerate}
    \item \textbf{环境准备}:安装 Python 3.10+ 及 \texttt{python3-tk}。
    \item \textbf{启动 UI}:运行 \texttt{python ui_main.py},记录显示的服务器 IP。
    \item \textbf{启动 Solver}:在同一局域网的机器(或本机)运行: \\
    \texttt{python solver_node.py --id 1 --host <IP>} \\
    \texttt{python solver_node.py --id 2 --host <IP>}
    \item \textbf{执行}:在 UI 界面点击“加载题目文件”或保持目录下的 \texttt{puzzle.txt} 为最新,系统将自动触发分布式计算过程。
\end{enumerate}

\section{分工情况}

\begin{itemize}
    \item \textbf{[陈卓尔]}:负责整体架构设计、TCP通信协议制定、common模块开发。
    \item \textbf{[罗竣泽]}:负责 UI 界面实现(Tkinter)、多线程服务器逻辑处理。
    \item \textbf{[胡枷浩]}:负责核心算法(IDA*、启发函数)的实现与性能测试。
\end{itemize}

\section{解决方案评价}

\subsection{题目要求完成情况}
\begin{itemize}
    \item 成功实现了局域网内的 1+2 分布式求解架构。
    \item UI界面精美,能实时展示不同 Solver 的交替移动过程。
    \item 算法高效,支持 $3 \times 3$ 和 $4 \times 4$ 甚至更高阶的解法检测。
\end{itemize}

\subsection{如何证明是共同完成而非“打酱油”}
在答辩时,我们提供以下证据:
\begin{enumerate}
    \item \textbf{底层日志}:UI 程序的日志中明确显示了来自不同 Solver ID 的消息包,且严格遵循交替逻辑。
    \item \textbf{状态依赖}:Solver 每次只发送一步指令,UI 执行后广播新状态。如果没有一方参与,游戏将陷入无限等待。
    \item \textbf{异构性}:支持 Solver 运行在不同的物理机器上,通过 IP 地址区分连接节点。
\end{enumerate}

\section{个人心得}

通过本次大作业,我们深度实践了分布式系统的设计理念。在处理华容道问题时,深刻体会到了 IDA* 算法在空间复杂度上的巨大优势。此外,多线程结合 Scoket 通信的调试过程也让我们更熟练地掌握了 Python 的高阶应用。分布式协作不仅是代码的拆分,更是逻辑的严密契合。

\end{document}
